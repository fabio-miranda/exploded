\section{Conclus�o}
\begin{frame}\frametitle{Conclus�o} 
\begin{itemize}
	\item A proposta de grafos de explos�o apresenta uma boa maneira para a visualiza��o de modelos 3D.
	\begin{itemize}
		\item A representa��o � uma forma ideal e intuitiva de relacionar diferentes partes e suas restri��es.
	\end{itemize}
	\item As principais caracter�sticas foram implementadas neste trabalho; alguns pontos foram simplificados, mas sem comprometer o resultado final.
\end{itemize}
\end{frame}


\begin{frame}\frametitle{Conclus�o} 
\begin{itemize}
	\item Um grande problema durante o desenvolvimento foi a falta de modelos 3D para o teste do sistema.
	\item Os modelos de teste certamente possuem uma complexidade bem inferior aos modelos 3D reais, seja na quantidade de partes ou na disposi��o das mesmas.
\end{itemize}
\end{frame}


\begin{frame}\frametitle{Conclus�o} 
\begin{itemize}
	\item Um problema que pode ser explorado � a explos�o de pe�as ao longo de um conjunto de eixos,
		\begin{itemize}
			\item Modelos mais complexos possam ser explodidos.
			\item A reconstru��o mental pode ficar comprometida (outros recursos gr�ficos poderiam ser estudados).
		\end{itemize}
	\item O uso de vista explodida para a visualiza��o de dados e informa��o (como \cite{1271635}), e n�o s� para modelos 3D convencionais, parece ser promissor.
	\item Unir diferentes t�cnicas de visualiza��o inteligente e avaliar os resultados e a facilidade de interpreta��o do modelo.
\end{itemize}
\end{frame}