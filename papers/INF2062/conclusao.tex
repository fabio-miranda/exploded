\section{Conclus�o e Trabalhos Futuros}
\label{conclusao}
As principais caracter�sticas propostas em \cite{1360700} foram implementadas neste trabalho. Algumas caracter�sticas menos importantes foram deixadas de lado, mas isso n�o comprometeu o resultado final atingido.

Um grande problema durante o desenvolvimento foi a falta de modelos 3D para o teste do sistema. Apesar dos modelos criados representarem bem as caracter�sticas e recursos do sistema, eles certamente possuem uma complexidade bem inferior aos modelos 3D reais, seja na quantidade de partes ou na disposi��o das mesmas.

Um problema que pode ser explorado � a explos�o de pe�as ao longo de mais de um eixo, possibilitando que modelos mais complexos sejam explodidos.

Na vis�o explodida, h� outros problemas que tamb�m podem ser explorados, al�m da disposi��o das pe�as. Como um dos principais objetivos da vista explodida � dar ao usu�rio um senso global das disposi��es das pe�as, certos \emph{features} gr�ficos podem ser utilizados, como linhas ligando a posi��o original da parte at� o seu destino.

Finalmente, um outro aspecto a ser explorado � a divis�o do modelo 3D em partes. O artigo de refer�ncia j� considera que o modelo de entrada est� bem dividido, algo que n�o ocorre necessariamente.


Conclus�o e trabalhos futuros
- Falar sobre vistas explodidas na vis�o de informa��o, como n�s da internet, etc.
- Falar da tese que utiliza realidade aumentada.